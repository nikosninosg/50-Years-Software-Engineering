\documentclass[11pt,a4paper,sans]{moderncv}      
\moderncvstyle{classic}                             
\moderncvcolor{green}                              
\usepackage[utf8]{inputenc}              
\usepackage[scale=0.75]{geometry}
\usepackage[english,greek]{babel}

\name{Νικόλαος}{Γεωργόπουλος-Νίνος}
\title{Βιογραφικό}                
\phone[mobile]{+30~(697)~395~796}   
                    
\extrainfo{ΑΜ:1054385}             
\photo[64pt][0.4pt]{PicProfil}                
\quote{Αρχή}                                
\begin{document}

\makecvtitle

\section{ΕΚΠΑΙΔΕΥΣΗ}
\cventry{2016--Σήμερα}{}{Μηχανικών Η/Υ και Πληροφορικής}{Πανεπιστήμιο Πατρών}{\textit{}}{}  
\cventry{2017--2018}{}{\foreignlanguage{english}{Mindspace}}{Πάτρα}{\textit{}}{-Όραμα: καινοτομία και επιχειρηματικότητα
\\	\foreignlanguage{english}{-Trainings for Managment}
\\	-Αρχηγός υποομάδας για την αυτοματοποίηση εργαστηρίου
\\	-Αρχηγός ομάδας για την Σχεδίαση Εργαστηρίου \\
	-Σχεδίαση και υλοποίηση καινοτόμου εργαστηρίου 
	\foreignlanguage{english}{(Innovation Lab)}
}


\section{ΕΜΠΕΙΡΙΑ}
\subsection{ΚΑΤΑΡΤΙΣΗ}
\cventry{}{Βασικές Αρχές του Ψηφιακού Μάρκετινγκ}{}{}{}{Πιστοποίηση}

\section{ΞΕΝΕΣ ΓΛΩΣΣΕΣ}
\cvitemwithcomment{Αγγλικά}{\foreignlanguage{english}{C2}}{}
\cvitemwithcomment{Γερμανικά}{Βασικός Χρήστης}{}

\section{ΙΚΑΝΟΤΗΤΕΣ}
\cvdoubleitem{\foreignlanguage{english}{Python Programmer}}{Ικανός Χρήστης}{\foreignlanguage{english}{IoT Developer}}{Βασικός Χρήστης}
\cvdoubleitem{Ασύρματα Δίκτυα Αισθητήρων}{Ικανός Χρήστης}{\foreignlanguage{english}{Linux}}{Ικανός Χρήστης}


\section{ΕΝΔΙΑΦΕΡΟΝΤΑ}
\cvitem{\foreignlanguage{english}{Raspberry Pi}}{Χρήση μικρουπολογιστών για την ασύρματη δικτώση συσκευών που αποσκοπούν στο έξυπνο σπίτι.}

\clearpage

\recipient{Company Recruitment team}{Company, Inc.\\123 somestreet\\some city}
\date{17, Απριλίου, 2019}

       


\end{document}



